\documentclass[12pt,a4paper]{article}
\usepackage[utf8x]{inputenc}
\usepackage[T1]{fontenc}
\usepackage{mathptmx}
\usepackage[pdftex]{graphicx}
\usepackage[pdftex,linkcolor=black,pdfborder={0 0 0}]{hyperref} % Format links for pdf
\usepackage{calc} % To reset the counter in the document after title page
\usepackage{enumitem} % Includes lists
\usepackage{caption}
\usepackage{subcaption}
\usepackage{amsmath}

\renewcommand\thesection{Task \arabic{section}}
\renewcommand\thesubsection{\alph{subsection}.)}
\renewcommand\thesubsubsection{\Roman{subsubsection}:}

\frenchspacing
\linespread{1.2}
\usepackage[a4paper, lmargin=0.12\paperwidth, rmargin=0.12\paperwidth, tmargin=0.05\paperheight, bmargin=0.1\paperheight]{geometry}

\usepackage[all]{nowidow} % Tries to remove widows
\usepackage[protrusion=true,expansion=true]{microtype}

\title{Excercise 1}
\author{Kai Schneider}
\date{\today}

\begin{document} 

\maketitle

\section{}

\subsection{}

$k = 2$, $\epsilon=0.5$ \newline
$\rightarrow P(\textrm{greedy})=1-\epsilon+\frac{\epsilon}{k}=1-0.5+\frac{0.5}{2}=0.75$ \newline
$\rightarrow P(\textrm{non-greedy})=\frac{\epsilon}{k}=\frac{0.5}{2}=0.25$

\subsection{}

$k=4 \; \rightarrow a_{i} \; \textrm{with} \; i=1:4$,  $\; Q_{1}(a_{i})=0$ \newline
with $A_{t}=\underset{a}{\operatorname{argmax}}Q_{t}(a)$ as the greedy policy and 
$Q_{t}(a)=\frac{\sum\limits_{i=1}^{t-1} R_{i,a_{i}=a}}{n(a)}$ and the given data:\newline

\begin{align*}
    A_{1}=1 \;\;\;\;\;\; & R_{1}=1 \\
    A_{2}=2 \;\;\;\;\;\; & R_{2}=1 \\
    A_{3}=2 \;\;\;\;\;\; & R_{3}=2 \\
    A_{4}=2 \;\;\;\;\;\; & R_{4}=2 \\
    A_{5}=3 \;\;\;\;\;\; & R_{5}=0
\end{align*}

\subsubsection{}

Step 1 (from $Q_{1}$ to $Q_{2}$) was definitely a random step because $Q_{1}(a_{i})=0 \; \forall i$, therefore the selection was arbitrary.

\begin{center}
\begin{tabular}{c |c c c c | c} 
    & $a_{1}$ & $a_{2}$ & $a_{3}$ & $a_{4}$ & action \\ [0.5ex] 
    \hline
    $Q_{1}$ & 0 & 0 & 0 & 0 & $A_{1}=1$ \\
    \hline
    $Q_{2}$ & 1 & 0 & 0 & 0 & $A_{2}=2$ \\ 
    \hline
    $Q_{3}$ & 1 & 1 & 0 & 0 & $A_{3}=2$ \\
    \hline
    $Q_{4}$ & 1 & 3 & 0 & 0 & $A_{4}=2$ \\
    \hline
    $Q_{5}$ & 1 & 5 & 0 & 0 & $A_{5}=3$ \\
    \hline
    $Q_{6}$ & 1 & 5 & 0 & 0 & - \\
    \hline
\end{tabular}
\end{center}

A random selection also has to be occured in step 2 ($Q_{2} \; \rightarrow \; Q_{3}$), because $A_{2}=2$ despite the 
$\underset{a}{\operatorname{argmax}}$
being $1$. Also in the fifth step ($Q_{5} \; \rightarrow \; Q_{6}$) action $A_{3}$ was selected, 
which had to be a random selection too.


\subsubsection{}

In general a random step could have occured at any other point too. 
Especially step 3 ($Q_{3} \; \rightarrow \; Q_{4}$) is a likely candidate, because the $argmax$ is either $1$ or $2$.
But even if the chosen $A_{i}$ is the $\underset{a}{\operatorname{argmax}}Q_{t}(a)$, it is still possible that this was a random selection. 






\end{document}