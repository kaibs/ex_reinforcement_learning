\documentclass[11pt,a4paper]{article}
\usepackage[utf8x]{inputenc}
\usepackage[T1]{fontenc}
\usepackage{mathptmx}
\usepackage{graphicx}
\usepackage[pdftex,linkcolor=black,pdfborder={0 0 0}]{hyperref} % Format links for pdf
\usepackage{calc} % To reset the counter in the document after title page
\usepackage{enumitem} % Includes lists
\usepackage{caption}
\captionsetup[figure]{font=small,labelfont=small,labelfont=bf}
\usepackage{subcaption}
\usepackage{amsmath}
\usepackage{amssymb}
\usepackage{amsfonts}
\usepackage{fancyvrb,newverbs,xcolor}
\usepackage{verbatim}
\definecolor{cverbbg}{gray}{0.93}

\newenvironment{lcverbatim}
 {\SaveVerbatim{cverb}}
 {\endSaveVerbatim
  \flushleft\fboxrule=0pt\fboxsep=.5em
  \colorbox{cverbbg}{%
    \makebox[\dimexpr\linewidth-2\fboxsep][l]{\BUseVerbatim{cverb}}%
  }
  \endflushleft
}

\renewcommand\thesection{Task \arabic{section}}
\renewcommand\thesubsection{\alph{subsection}.)}
\renewcommand\thesubsubsection{\Roman{subsubsection}:}

\frenchspacing
\linespread{1.2}
\usepackage[a4paper, lmargin=0.12\paperwidth, rmargin=0.12\paperwidth, tmargin=0.05\paperheight, bmargin=0.1\paperheight]{geometry}

\usepackage[all]{nowidow} % Tries to remove widows
\usepackage[protrusion=true,expansion=true]{microtype}

\title{Exercise 10}
\author{Kai Schneider}
\date{\today}

\begin{document} 

\maketitle

\section{DQN on the Cart-Pole}

\subsection{}

\begin{itemize}
  \item NNs tend to diverge (due to high correlation states and actions, NOT i.i.d) \\ $\rightarrow$
        Experience Replay (sampling training data from buffer of past experiences)
  \item oscillation due to big changes in Q-values \\ $\rightarrow$
        much lower learning rate than in other DL-scenarios \\
        $\rightarrow$ reward clipping, normalize $[0,1]$
  \item bias introduced at beginning of training and instability due to similar subsequend steps \\ $\rightarrow$ 
        2 NNs. Predictions of a target NN are used to get the Q-values during training.
        This target network is only synced periodically. The main NN then uses this to backprpagate its weights.
\end{itemize}





\end{document}

