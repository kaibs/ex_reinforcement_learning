\documentclass[11pt,a4paper]{article}
\usepackage[utf8x]{inputenc}
\usepackage[T1]{fontenc}
\usepackage{mathptmx}
\usepackage{graphicx}
\usepackage[pdftex,linkcolor=black,pdfborder={0 0 0}]{hyperref} % Format links for pdf
\usepackage{calc} % To reset the counter in the document after title page
\usepackage{enumitem} % Includes lists
\usepackage{caption}
\captionsetup[figure]{font=small,labelfont=small,labelfont=bf}
\usepackage{subcaption}
\usepackage{amsmath}
\usepackage{amssymb}
\usepackage{amsfonts}
\usepackage{fancyvrb,newverbs,xcolor}
\usepackage{verbatim}
\definecolor{cverbbg}{gray}{0.93}

\newenvironment{lcverbatim}
 {\SaveVerbatim{cverb}}
 {\endSaveVerbatim
  \flushleft\fboxrule=0pt\fboxsep=.5em
  \colorbox{cverbbg}{%
    \makebox[\dimexpr\linewidth-2\fboxsep][l]{\BUseVerbatim{cverb}}%
  }
  \endflushleft
}

\renewcommand\thesection{Task \arabic{section}}
\renewcommand\thesubsection{\alph{subsection}.)}
\renewcommand\thesubsubsection{\Roman{subsubsection}:}

\frenchspacing
\linespread{1.2}
\usepackage[a4paper, lmargin=0.12\paperwidth, rmargin=0.12\paperwidth, tmargin=0.05\paperheight, bmargin=0.1\paperheight]{geometry}

\usepackage[all]{nowidow} % Tries to remove widows
\usepackage[protrusion=true,expansion=true]{microtype}

\title{Exercise 7}
\author{Kai Schneider}
\date{\today}

\begin{document} 

\maketitle

\section{Linear function approximation}

\subsection{}

With linear function approximation we have a value function with the following form:\\
$ \hat{\upsilon}(s,w)=w^{T}x(s)=\sum_{i=1}^{d}w_{i}x_{i}(s)$\\
For tabular methods we can write all $x_{i}(s)$ as 
$x_{i}(s)=[0 \dots \underbrace{v(s)}_{\text{i-th entry}}\dots 0]^T$\\\\
This way our value function can be written the same as for the linear function 
approximating with the weight vector $w = [1 \dots 1]$.\\
This way the tabular version is only a special case of the linear function 
approximation with the $dim(w)=|S|$.

\subsection{}

Sarsa: $Q(S_t,A_t) \leftarrow Q(S_t,A_t) + \alpha \delta_{t}$ with  
$\delta_{t}=[R_{t+1}+\gamma Q(S_{t+1},A_{t+1})-Q(S_t,A_t)]$

\subsubsection{tabular}

For the tabular case we can directly use the formula given in the slides (v05s15):
\vspace{10pt}
\flushleft
$Q(S_t,A_t) \leftarrow Q(S_t,A_t) + \alpha [R_{t+1}+\gamma Q(S_{t+1},A_{t+1})-Q(S_t,A_t)]$

\subsubsection{function approximation}

For state-action values we have the following update rule for weights in control:
\vspace{10pt}
\flushleft
$w_{t+1} \leftarrow w_{t} + \alpha [U_{t} - \hat{q}(S_t,A_t,w)]\nabla\hat{q}(S_t,A_t,w)$\\
with $U_t = R_{t+1}+\gamma \hat{q} (S_{t+1},A_{t+1},w)$ for one-step SARSA\\
\vspace{10pt}
plugged in we get: $w_{t+1} \leftarrow w_{t} + \alpha [R_{t+1}+\gamma \hat{q} (S_{t+1},A_{t+1},w) - \hat{q}(S_t,A_t,w)]\nabla\hat{q}(S_t,A_t,w)$

\subsubsection{linear function approximation}

With the value function beeing $\hat{v}(s,w)=w^T x(s)$ for the linear case and the update formula
for the linear TD $w \leftarrow w + \alpha [R_{t+1}+\gamma w^T x(S_{t+1}) - w^T x(S_t)]x(S_t)$ we can get:\\
\vspace{10pt}
\flushleft
$w_{t+1} \leftarrow w_{t} + \alpha [R_{t+1}+\gamma w^T x(S_{t+1},A_{t+1}) - w^T x(S_t,A_t)]x(S_t,A_t)$







\end{document}

